\documentclass{article}

\usepackage{amsmath}
\usepackage[utf8]{inputenc}
\usepackage{mathabx}
\usepackage{amsthm}
\usepackage{enumitem}
\usepackage{bm}

\theoremstyle{definition} % by default it is "plain" which will italicize the body
\newtheorem{theorem}{Theorem}[section] % will be numbered section.theorem

\theoremstyle{definition}
\newtheorem{corollary}{Corollary}[theorem] % will be numbered section.theorem.corollary

\theoremstyle{definition}
\newtheorem{lemma}{Lemma}[section]

% \newtheorem{lemma}[theorem]{Lemma} % will follow the numbering as if it is a theorem.
% So after theorem 1.2 comes lemma 1.3

\theoremstyle{definition}
\newtheorem{definition}{Definition}[section]

\counterwithin*{equation}{section}

\title{Summary for ``Elementary Number Theory: Second Edition by Underwood Dudley"}
\date{2019-07-08}
\author{Agro Rachmatullah}

\begin{document}
  \pagenumbering{gobble}
  \maketitle
  
  \newpage
  \pagenumbering{arabic}
  \section{Integers}
  
  \begin{definition}[Least-integer principle]
  A nonempty set of integers that is bounded below contains a smallest element.
  \end{definition}
  
  \subparagraph{Example} The set $\{4, 5, 6\}$ has 4 as the smallest element. The set $\{10, 12, 14, \dots\}$ has 10 as the smallest element.
  
  \begin{definition}[Greatest-integer principle]
  A nonempty set of integers that is bounded above contains a largest element.
  \end{definition}
  
  \subparagraph{Example} The set $\{4, 5, 6\}$ has 6 as the largest element. The set $\{1\}$ has 1 as the largest element.
  
  \begin{definition}
  $a$ divides $b$ (written $a \divides b$) if and only if there is an integer $d$ such that $ad = b$.
  
  \subparagraph{Examples} $3 \divides 6$, $15 \divides 60$, $9 \divides 9$, $-4 \divides 16$, and $2 \divides -100$.
  \end{definition}
  
  \begin{definition}
  If $a$ does not divide $b$, we write $a \notdivides b$.
  
  \subparagraph{Examples} $10 \notdivides 5$ and $3 \notdivides 7$.
  \end{definition}
  
  \begin{lemma}
    If $d \divides a$ and $d \divides b$, then $d \divides (a + b)$.
  \end{lemma}
  
  \subparagraph{Example} $2 \divides 4$ and $2 \divides 10$, so $2 \divides 14$.
  
  \begin{lemma}
    If $d \divides a_1$, $d \divides a_2$, $\dots$ $d \divides a_n$, then $d \divides (c_1a_1 + c_2a_2 + \dots + c_na_n)$ for any integers $c_1, c_2, \dots, c_n$
  \end{lemma}
  
  \subparagraph{Example} $2 \cdot 6 + 4 \cdot 9 = 12 + 36 = 48$. Because $3 \divides 6$ and $3 \divides 9$, we conclude that $3 \divides 48$.
  
  \begin{definition}
    $d$ is the greatest common divisor of $a$ and $b$ (written $d = (a, b)$) if and only if
    
    \begin{enumerate}[label=(\roman*)]
      \item $d \divides a$ and $d \divides b$, and
    
      \item if $c \divides a$ and $c \divides b$, then $c \le d$
    \end{enumerate}
  \end{definition}
  
  \subparagraph{Examples} $(2, 6) = 2$ and $(5, 7) = 1$.
  
  \begin{theorem}
    If $(a, b) = d$, then $(a / d, b / d) = 1$.
  \end{theorem}
  
  \subparagraph{Examples} \ 
  
  $(16, 20) = 4$, so $(16 / 4, 20 / 4) = (4, 5) = 1$
  
  $(12, 6) = 3$, so $(12 / 3, 6 / 3) = (4, 2) = 2$
  
  \begin{proof}
    Suppose that $c = (a / d, b / d)$. If follows that $c \divides (a / d)$ and $c \divides (b / d)$. Therefore there are
    integers $q$ and $r$ such that $cq = a / d$ and $cr = b / d$. That is,
    
      \begin{equation*}
        (cd)q = a \;\;\text{and}\;\; (cd)r = b
      \end{equation*}
    
    which means $cd$ is a divisor of both $a$ and $b$. Because $(a, b) = d$, it must be the case that $cd \leq d$.
    $d$ is positive so $c \leq 1$.
    
    Because $c = (a / d, b / d)$, it follows that $c \geq 1$. Therefore $c = 1$.
  \end{proof}
  
  \begin{definition}
    If $(a, b) = 1$, then we will say that $a$ and $b$ are \textbf{relatively prime}.
  \end{definition}
  
  \subparagraph{Examples} $(4, 5) = 1$, so 4 and 5 are relatively prime. 10 and 7 are also relatively prime.
  
  \begin{theorem}[The Division Algorithm]
    \label{th:division_algorithm} Given positive integers $a$ and $b$, $b \neq 0$, there exist unique integers
    $q$ and $r$, with $0 \leq r < b$ such that
    
    \begin{equation*}
      a = bq + r
    \end{equation*}
  \end{theorem}
  
  \subparagraph{Example} With $a = 17$ and $b = 5$, we have $17 = 5 \cdot 3 + 2$
  
  \begin{proof}
    Consider the set of integers $\{a, a - b, a - 2b, a - 3b, \dots, a - qb\}$ bounded below by 0.
    It contains members that are nonnegative and nonempty (because at least $a$ is a member).
    From the least-integer principle, it contains a smallest element $a - qb$.
    
    The smallest element must be less than $b$, because if not the smallest element in the set would have
    to be $a - (q + 1)b$.
    
    Ler $r = a - qb$. It follows that $a = bq + r$ and we only have to show that $q$ and $r$ are unique.
    
    Suppose that we have found $q$, $r$ and $q_1$, $r_1$ such that $a = bq + r = bq_1 + r_1$ with
    $0 \leq r < b$ and $0 \leq r_1 < b$. Subtracting, we get
    
    \begin{align*}
      0 &= b(q - q_1) + (r - r_1) \\
      b(q_1 - q) &= r - r_1
    \end{align*}
    
    Since $b$ divides the left side of the equation, it follows that $b \divides r - r_1$.
    
    Because $0 \leq r_1 < b$, we have $-b < -r_1 \leq 0$. We also have $0 \leq r < b$, so it follows that
    
    \begin{equation*}
      -b < r - r_1 < b
    \end{equation*}
    
    Since the only number in that range divisible by $b$ is 0, $r - r_1 = 0$ which implies $q - q_1 = 0$.
    Hence the numbers $q$ and $r$ in the theorem is unique.
  \end{proof}
  
  \begin{lemma}
    If $a = bq + r$, then $(a, b) = (b, r)$.
  \end{lemma}
  
  \begin{proof}
    Let $d = (a, b)$. Because $d \divides a$ and $d \divides b$, we know from $a = bq + r$ that $d \divides r$.
    Therefore, $d$ is a common divisor of $b$ and $r$. It remains to show that $d$ is not just any common
    divisor but in fact the greatest common divisor.
    
    Now let us assume that $c$ is a common divisor of $b$ and $r$, so $c \divides b$ and $c \divides r$.
    From the equation $a = bq + r$, we know that $c \divides a$. So $c$ is common divisor of both
    $a$ and $b$. Because $(a, b) = d$, it must be the case that $c \leq d$.
    
    Since $d$ is a common divisor of $b$ and $r$, and for any common divisor $c$ we have $c \leq d$,
    we have proven that $(b, r) = d$.
  \end{proof}
  
  \begin{theorem}[The Eucledian Algorithm]
    If $a$ and $b$ are positive integers, $b \ne 0$, and
    
    \begin{align*}
      a &= bq + r,         &0 &\leq r < b, \\
      b &= rq_1 + r_1, &0 &\leq r_1 < r, \\
      r &= r_1q_2 + r_2, &0 &\leq r_2 < r_1 \\
      \cdot & &. \\
      \cdot & &. \\
      \cdot & &. \\
      r_k &= r_{k+1}q_{k+2} + r_{k+2}, &0 &\leq r_{k+2} < r_{k+1}
    \end{align*}
    
    then for $k$ large enough, say $k = t - 1$, we have
    
    \begin{equation*}
      r_{t - 1} = r_tq_{t+1}
    \end{equation*}
    
    and $(a, b) = r_t$.
  \end{theorem}
  
  \begin{proof}
    The sequence
    
    \begin{equation*}
      b > r > r_1 > r_2 > \dots
    \end{equation*}
    
    is decreasing, and we know that they are nonnegative, so we will eventually reach 0. Suppose $r_{t+1} = 0$.
    Then we have $r_{t - 1} = r_tq_{t+1}$. If we apply Lemma 3 over and over,
    
    \begin{equation*}
      (a, b) = (b, r) = (r, r_1) = (r_1, r_2) = \dots = (r_{t-1}, r_t) = r_t
    \end{equation*}
  \end{proof}
  
  \begin{theorem}
    If $(a, b) = d$, then there are integers $x$ and $y$ such that
    
    \begin{equation*}
      ax + by = d
    \end{equation*}
    \label{th:gcd_to_linear}
  \end{theorem}
  
  \begin{proof}
    Let us assume that $a$ and $b$ are positive integers with $a \geq b$ and $b \neq 0$.
    We can always switch the order of $a$ and $b$, and if $b = 0$ then the proof is trivial.
    
    If $(a, b) = b$, then $a \cdot 0 + b \cdot 1 = b$ so the equation is true with $x = 0$ and $y = 1$.
    
    For $d < b$, then $d$ will be one of the remainders in the set of equations from Theorem 3.
    If we call the remainders $r_0, r_1, \dots$ then we can rewrite the equations as
    
    \begin{align*}
      r_0 &= a - bq \\
      r_1 &= b - r_0q_1 \\
      r_2 &= r_0 - r_1q_2 \\
      &\dots \\
      r_n &= r_{n-2} - r_{n-1}q_n
    \end{align*}
    
    For the base case of $r_0$ and $r_1$, it is easy to confirm that they can be written as
    $ax + by$.
    
    Now, assuming that $r_{n-2} = ax + by$ and $r_{n-1} = ax' + by'$, then
    
    \begin{align*}
      r_n &= r_{n-2} - r_{n-1}q_n \\
      &= ax + by - q_n(ax' + by') \\
      &= a(x - q_nx') + b(y - q_ny')
    \end{align*}
    
    Because the base case and inductive case is proven, it is proved for all $r_n$.
    
    If one or both of $a$ and $b$ are negative, we can use the property $(a, b) = (-a, b) = (a, -b) = (-a, -b)$.
    We can also switch the order such that $a \geq b$ as required by the beginning of the proof.
  \end{proof}
  
  \begin{corollary}
    If $d \divides ab$ and $(d, a) = 1$, then $d \divides b$.
    \label{cor:must_divide_one_of_them}
  \end{corollary}
  
  \begin{proof}
    Because $d$ and $a$ is relatively prime, we have
    
    \begin{align*}
      dx + ay &= 1 \\
      d(bx) + (ab)y &= b
    \end{align*}
    
    Because the left side is divisible by $d$, we conclude that $d \divides b$.
  \end{proof}
  
  \begin{corollary}
    Let $(a, b) = d$, and suppose that $c \divides a$ and $c \divides b$.
    Then $c \divides d$.
  \end{corollary}
  
  \subparagraph{Examples} $(18, 12) = 6$, and 3 is a common divisor
  of both 18 and 12. Thus by the corollary $3 \divides 6$.
  
  \begin{proof}
    We know that there are integers $x$ and $y$ such that
    
    \begin{equation*}
      ax + by = d
    \end{equation*}
    
    Because $c \divides ax$ and $c \divides by$, $c$ divides the right hand side too.
  \end{proof}
  
  \begin{corollary}
    If $a \divides m$, $b \divides m$, and $(a, b) = 1$, then $ab \divides m$.
  \end{corollary}
  
  \subparagraph{Examples} $3 \divides 30$, $5 \divides 30$, and $(3, 5) = 1$. Thus $3 \cdot 5 = 15 \divides 30$.
  
  \begin{proof}
    $b \divides m$ means there is an integer $q$ such that $m = bq$.
    Since $a \divides m$, we have $a \divides bq$.
    
    However since $(a, b) = 1$, from Corollary 1 we know that $a \divides q$. Therefore
    there is an integer $r$ such that $q = ar$, so $m = bar = (ab) r$. Thus $ab \divides m$.
  \end{proof}
  
  \section{Unique Factorization}
  
  \begin{definition}
    A \textbf{prime} is an integer that is greater than 1 and has no positive divisors other than 1 and itself.
  \end{definition}
  
  \subparagraph{Examples} 2, 3, 5, 7 and 11 are primes.
  
  \begin{definition}
    An integer that is greater than 1 but is not prime is called \textbf{composite}.
  \end{definition}
  
  \subparagraph{Examples} 4 is a composite because it is divisible by 2. 10 is a composite because it
  has 2 and 5 as its divisor.
  
  \begin{definition}
    1 is neither a prime nor composite. We will call 1 a \textbf{unit}.
  \end{definition}
  
  \begin{lemma}
    Every integer $n$, $n > 1$, is divisible by a prime.
    \label{lem:n_is_divisible_by_prime}
  \end{lemma}
  
  \begin{proof}
    Consider the set of all divisors of $n$ larger than 1 and smaller than $n$ itself. If it is empty,
    then $n$ is a prime which means that it is divisible by a prime (namely itself).
    
    If it is nonempty, then by the least integer principle it has a smallest divisor, say $p$. If $p$
    is not a prime, then it has divisor $q > 1$ but smaller than itself. However $q$ must divide
    $n$ which is a contradiction because $p$ is supposed to be the smallest in the set. Therefore
    $p$ is a prime, and $n$ has a prime divisor which is $p$.
  \end{proof}
  
  \begin{proof}
    (By induction) The lemma is true by inspection for $n = 2$. Suppose it is true for $n \leq k$. Then either
    $k + 1$ is prime, in which case we are done, or it is divisible by some number $k_1$ with
    $k_1 \leq k$. But from the induction assumption, $k_1$ is divisible by a prime, and this prime
    also divides $k + 1$. Again, we are done.
  \end{proof}
  
  \begin{lemma}
    Every integer $n$, $n > 1$, can be written as a product of primes.
    \label{lem:integer_is_a_product_of_primes}
  \end{lemma}
  
  \begin{proof}
    From Lemma 1, we know that there is a prime $p_1$ such that $p_1 \divides n$. That is, $n = p_1n_1$,
    where $1 \leq n_1 < n$. If $n_1 = 1$, then we are done: $n = p_1$ is an expression of $n$ as a products of
    primes. If $n_1 > 1$, then from Lemma 1 again, there is a prime that divides $n_1$. That is, $n_1 = p_2n_2$,
    where $p_2$ is a prime and $1 \leq n_2 < n_1$. If $n_2 = 1$, again we are done: $n = p_1p_2$ is written as a 
    product of primes. But if $n_2 > 1$, then Lemma 1 once again says that $n_2 = p_3n_3$, with $p_3$ a prime
    and $1 \leq n_3 < n_2$. If $n_3 = 1$, we are done. If not we continue. We will sooner or later come to one of the
    $n_i$ equal to 1, because $n > n_1 > n_2 > \dots$ and each $n$, is positive; such a sequence cannot continue forever.
    For some $k$, we will have $n_k = 1$, in which case $n = p_1p_2 \cdots p_k$ is the desired expression of n as
    a product of primes. Note that the same prime may occur several times in the product.
  \end{proof}
  
  \begin{theorem}[Euclid]
    There are infinitely many primes.
  \end{theorem}
  
  \begin{proof}
    Suppose not. Then there are only finitely many primes. Denote them by $p_1, p_2, \dots, p_r$. Consider
    the integer
    
    \begin{equation}
      n = p_1p_2\cdots p_r + 1
      \label{eq:new_prime}
    \end{equation}
    
    From Lemma 1, we see that $n$ is divisible by a prime, and since there are only finitely many primes, it
    must be one of $p_1, p_2, \dots, p_r$. Suppose that it is $p_k$. Then since
    
    \begin{equation*}
      p_k \divides n \;\;\text{and}\;\; p_k \divides p_1p_2\cdots p_r\;\text{,}
    \end{equation*}
    
    it divides two of the terms in (\ref{eq:new_prime}). Consequently it divides the other term in (\ref{eq:new_prime}); thus $p_k \divides 1$.
    This is nonsense: no primes divide 1 because all are greater than 1. This contradiction shows that
    we started with an incorrect assumption. Since there cannot be only finitely many primes, there
    are infinitely many.
  \end{proof}
  
  \begin{lemma}
    If $n$ is composite, then it has divisor $d$ such that $1 < d \leq n^\frac{1}{2}$.
    \label{lem:has_divisor_leq_sqrt}
  \end{lemma}
  
  \begin{proof}
    Since $n$ is composite, there are integers $d_1$ and $d_2$ such that $d_1d_2 = n$
    and $1 < d_1 < n$, $1 < d_2 < n$. If $d_1$ and $d_2$ are both larger than $n^\frac{1}{2}$,
    then
    
    \begin{equation*}
      n = d_1d_2 > n^\frac{1}{2} n^\frac{1}{2} = n
    \end{equation*}
  \end{proof}
  
  which is impossible. Thus, one of $d_1$ and $d_2$ must be less than or equal to $n^\frac{1}{2}$.
  
  \begin{lemma}
    If $n$ is composite, then it has a prime divisor less than or equal to $n^\frac{1}{2}$.
  \end{lemma}
  
  \begin{proof}
    We know from Lemma \ref{lem:has_divisor_leq_sqrt} that $n$ has a divisor---call it d---such that $1 < d \leq n^\frac{1}{2}$.
    From Lemma \ref{lem:n_is_divisible_by_prime}, we know that $d$ has a prime divisor $p$. Since $p \leq d \leq n^\frac{1}{2}$,
    the lemma is proved.
  \end{proof}
  
  \begin{lemma}
    If $p$ is a prime and $p \divides ab$, then $p \divides a$ or $b \divides b$.
    \label{lem:prime_must_divide_one_of_two}
  \end{lemma}
  
  \subparagraph{Examples} $2 \divides 4 \cdot 3$, so 2 must either divide 4 or 3.
  Indeed, $2 \divides 4$.
  
  \subparagraph{} The same couldn't be said if the divisor is not a prime. For example, even though
  $4 \divides 2 \cdot 6$, 4 doesn't divide either 2 nor 6.
  
  \begin{proof}
    Since $p$ is prime, its only positive divisors are 1 and $p$. Thus $(p, a) = p$ or
    $(p, a) = 1$. In the first case, $p \divides \bm{a}$, and we are done. In the second case,
    Corollary \ref{cor:must_divide_one_of_them} tells us that $p \divides \bm{b}$, and again we are done.
  \end{proof}
  
  \begin{lemma}
    If $p$ is a prime and $p \divides a_1a_2\cdots a_k$, then $p \divides a_i$ for some $i$, $i = 1, 2, \dots, k$.
    \label{lem:prime_must_divide_one_of_them_k}
  \end{lemma}
  
  \begin{proof}
    Lemma \ref{lem:prime_must_divide_one_of_them_k} is true by inspection if $k = 1$, and
    Lemma \ref{lem:prime_must_divide_one_of_two} shows that it is true if $k = 2$. We will proceed by induction.
    Suppose that Lemma \ref{lem:prime_must_divide_one_of_them_k} is true for $k = r$. Suppose that
    $p \divides a_1a_2\cdots a_{r+1}$. The $p \divides a_1a_2\cdots a_r$ or $p \divides a_{r + 1}$. In the
    first case, the induction assumption tells us that $p \divides a_i$ for some $i$, $i = 1, 2, \dots, r$. In the
    second case, $p \divides a_i$ for $i = r + 1$. In either case, $p \divides a_i$ for some $i$, $i =1, 2, \dots, r + 1$.
    Thus, if the lemma is true for $k = r$, it is true for $k = r + 1$, and since it is true for $k = 1$ and $k = 2$, it is
    true for any positive integer $k$.
  \end{proof}
  
  \begin{lemma}
    If $p, q_1, q_2, \dots, q_n$ are primes, and $p \divides q_1q_2\cdots q_n$, then $p = q_k$ for some $k$.
    \label{lem:prime_must_be_one_of_them}
  \end{lemma}
  
  \begin{proof}
    From Lemma \ref{lem:prime_must_divide_one_of_them_k} we know that $p \divides q_k$ for some $k$.
    Since $p$ and $q_k$ are primes, $p = q_k$. (The only positive divisors of $q_k$ are 1 and $q_k$, and
    $p$ is not 1.)
  \end{proof}
  
  \begin{theorem}[The Unique Factorization Theorem]
    Any positive integer can be written as a product of primes in one and only one way.
  \end{theorem}
  
  \begin{proof}
    Recall that we agreed to consider as identical all factorizations that differ only in the order of the factors.
    
    We know already from Lemma \ref{lem:integer_is_a_product_of_primes} that any integer $n$, $n > 1$
    can be written as a product of primes. Thus to complete the proof of the theorem, we need to show
    that $n$ cannot have two such representations. That is, if
    
    \begin{equation}
      n = p_1p_2\cdots p_m \;\;\text{and}\;\; n = q_1q_2\cdots q_r
      \label{eq:different_prime_factors}
    \end{equation}
    
    then we must show that the same primes appear in each product, and the same number of times,
    though their order may be different. That is, we must show that the integer $p_1, p_2, \dots, p_m$
    are just a rearrangement of the integers $q_1q_2, \dots, q_r$. From (\ref{eq:different_prime_factors})
    we see that since $p_1 \divides n$,
    
    \begin{equation*}
      p_1 \divides q_1q_2\cdots q_r
    \end{equation*}
    
    From Lemma \ref{lem:prime_must_be_one_of_them}, it follows that $p_1 = q_i$ for some $i$.
    If we divide
    
    \begin{equation*}
      p_1p_2\cdots p_m = q_1q_2\cdots q_r
    \end{equation*}
    
    by the common factor, we have
    
    \begin{equation}
      p_2p_3\cdots p_m = q_1q_2\cdots q_{i-1}q_{i+1}\cdots q_r\;\text{.}
      \label{eq:factorization_divided_once}
    \end{equation}
    
    Because $p_2$ divides the left-hand side of this equation, it also divides the right-hand side.
    Applying Lemma \ref{lem:prime_must_be_one_of_them} again, it follows that $p_2 = q_j$
    for some $j$ ($j = 1, 2, \dots, i - 1, i + 1, \dotsm n$). Cancel this factor from both sides of
    (\ref{eq:factorization_divided_once}), and continue the process. Eventually we will find that
    each $p$ is a $q$. We cannot run out of $q$'s before all the $p$'s are gone, because we would
    then have a product of primes equal to 1, which is impossible. If we repeat the argument with
    the $p$'s and $q$'s interchanged, we see that each $q$ is a $p$. Thus the numbers $p_1,
    p_2, \dots, p_m$ are rearrangement of $q_1, q_2, \dots q_r$ and the two factorizations differ only in the
    order of the factors.
  \end{proof}
  
  \begin{proof}
    (Induction) The theorem is true, by inspection, for $n = 2$. Suppose that it is true for $n \leq k$.
    Suppose that $k + 1$ has two representations:
    
    \begin{equation*}
      k + 1 = p_1p_2\cdots p_m = q_1q_2\cdots q_r\;\text{.}
    \end{equation*}
    
    As in the last proof, $p_1 = q_i$ for some $i$, so
    
    \begin{equation*}
      p_2p_3\cdots p_m = q_1q_2\cdots q_{i-1}q_{i+1}\cdots q_r\;\text{.}
    \end{equation*}
    
    But this number is less than or equal to $k$, and by the induction assumption, its prime decomposition
    is unique. Hence the integers $q_1, q_2, \dots, q_{i-1}, q_{i + 1}, \dots, q_r$ are a rearrangement of
    $p_2, p_3, \dots, p_m$, and since $p_1 = q_i$ the proof is complete.
  \end{proof}
  
  \begin{definition}
    From the unique factorization theorem it follows that each positive integer can be written in exactly
    one way in the form
    
    \begin{equation*}
      n = p_1^{e_1}p_2^{e_2}\cdots p_k^{e_k}
    \end{equation*}
    
     where $e_i \geq 1, i = 1, 2, \dots, k$, each $p_i$ is a prime, and $p_i \neq p_j$ for $i \neq j$.
     We call this representation the \textbf{prime-power decomposition} of $n$,
  \end{definition}
  
  \begin{theorem}
    If $e_1 \geq 0$, $f_1 \geq 0$, $(i = 1, 2, \dots k)$,
    
    \begin{equation*}
      m = p_1^{e_1}p_2^{e_2}\cdots p_k^{e_k} \;\;\text{and}\;\; n = p_1^{f_1}p_2^{f_2}\cdots p_k^{f_k}\;\text{,}
    \end{equation*}
    
    then
    
    \begin{equation*}
      (m, n) = p_1^{g_1}p_2^{g_2}\cdots p_k^{g_k}
    \end{equation*}
    
    where $g = \text{min}(e_i,f_i)$, $i = 1, 2, \dots k$
  \end{theorem}
  
  \section{Linear Diophantine Equations}
  
  \begin{definition}
    Equations in which we look for solutions in a restricted class of numbers\textemdash be they positive
    integers, negative integers, rational numbers, or whatever\textemdash are called \textbf{diophantine
    equations}.
  \end{definition}
  
  \subparagraph{Example} $x^2 + y^2 = z^2$ where we look for solutions in integers. One such solution
  is $x = 3$, $y = 4$, $z = 5$.
  
  \begin{lemma}
    If $x_0, y_0$ is a solutions of $ax + by = c$, then so is
    
    \begin{equation*}
      x_0 + bt, y_0 - at
    \end{equation*}
    
    for any integer $t$.
  \end{lemma}
  
  \subparagraph{Example} $x + 2y = 4$ has $x = 0, y = 2$ as one of its solution. If we set $t = 2$, then
  
  \begin{align*}
    x &= 0 + 2 \cdot 2 = 4 \\
    y &= 2 - 1 \cdot 2 = 0
  \end{align*}
  
  is also a solution for the diophantine equation.
  
  \begin{proof}
    We are given that $ax_0 + by_0 = c$. Thus
    
    \begin{align*}
      a(x_0 + bt) + b(y_0 - at) &= ax_0 + abt + by_0 - bat \\
      &= ax_0 + by_0 \\
      &= c
    \end{align*}
    
    so $x_0 + bt, y_0 - at$ satisfies the equation too.
  \end{proof}
  
  \begin{lemma}
    If $(a, b) \not\divides c$. then $ax + by = c$ has no solutions, and if $(a, b) \divides c$, then
    $ax + by = c$ has a solution.
  \end{lemma}
  
  \subparagraph{Example} The equation $2x + 4y = 5$ has no solution because $(2, 4) = 2 \not\divides 5$.
  
  \begin{proof}
    Suppose that there are integers $x_0, y_0$ such that $ax_0 + by_0 = c$, Since $(a, b) \divides ax_0$ and
    $(a, b) \divides by_0$, it follows that $(a, b) \divides c$. Conversely, suppose that $(a, b) \divides c$.
    Then $c = m(a, b)$ for some $m$. From Theorem \ref{th:gcd_to_linear}, we know that there are integers $r$ and $s$ such that
    
    \begin{equation*}
      ar + bs = (a, b)
    \end{equation*}
    
    Then
    
    \begin{equation*}
      a(rm) + b(sm) = m(a, b) = c
    \end{equation*}
    
    and $x = rm, y = sm$ is a solution.
  \end{proof}
  
  \begin{lemma}
    Suppose that $(a, b) = 1$ and $x_0, y_0$ is a solution of $ax + by = c$. Then all solutions of $ax + by = c$ are
    given by
    
    \begin{align*}
      x &= x_0 + bt \\
      y &= y_0 - at
    \end{align*}
    
    where $t$ is an integer.
  \end{lemma}
  
  \begin{proof}
    We see from Lemma 2 that the equation does have a solution, because $(a, b) = 1$ and $1 \divides c$ for all
    $c$. Then, let $r, s$ be \textit{any} solution of $ax + by = c$. We want to show that $r = x_0 + bt$ and
    $s = y_0 - at$ for some integer $t$. From $ax_0 + by_0 = c$ follows
    
    \begin{equation*}
      c - c = (ax_0 + by_0) - (ar + bs)
    \end{equation*}
    
    or
    
    \begin{equation}
      a(x_0 - r) + b(y_0 - s) = 0
      \label{eq:0697C052_6EA0_41C8_B3CB_348A6DA20C28}
    \end{equation}
    
    Because $a \divides a(x_0 - r)$ and $a \divides 0$, we have $a \divides b(y_0 - s)$. But we have supposed that
    $a$ and $b$ are relatively prime. It follows from Corollary 1.1 that $a \divides y_0 - s$. That is, there is an integer
    $t$ such that
    
    \begin{equation}
      at = y_0 - s
      \label{eq:4229D1AA_9738_411C_AC87_33474E7DF1D4}
    \end{equation}
    
    Substituting in (\ref{eq:0697C052_6EA0_41C8_B3CB_348A6DA20C28}), this gives
    
    \begin{equation*}
      a(x_0 - r) + bat = 0
    \end{equation*}
    
    Because $a \neq 0$, we may cancel it to get
    
    \begin{equation}
      x_0 - r + bt = 0
      \label{eq:2B9C5726_F9C3_49DA_A9D6_6B42D1E4246D}
    \end{equation}
    
    But (\ref{eq:4229D1AA_9738_411C_AC87_33474E7DF1D4}) and (\ref{eq:2B9C5726_F9C3_49DA_A9D6_6B42D1E4246D}) say that
    
    \begin{align*}
      s &= y_0 - at \\
      r &= x_0 + bt
    \end{align*}
    
    Since $r, s$ was \textit{any} solution, the lemma is proved.
  \end{proof}
  
  \begin{theorem}
    The linear diophantine equation $ax + by = c$ has no solutions if $(a, b) \not\divides c$. If $(a, b) \divides c$, there are
    infinitely many solutions
    
    \begin{equation*}
      x = r + \frac{b}{(a, b)}t,\;\;\;y = s - \frac{a}{(a, b)}t
    \end{equation*}
    
    where $r$, $s$ is any solution and $t$ is an integer.
  \end{theorem}
  
  \section{Congruences}
  
  \begin{definition}
    We say that $a$ is \textbf{congruent} to $b$ modulo $m$ (in symbols, $a \equiv b \pmod{m}$) if and only if $m \divides (a - b)$
    and we will suppose always that $m > 0$.
  \end{definition}
  
  \begin{theorem}
    $a \equiv b \pmod{b}$ if and only if there is an integer $k$ such that $a = b + km$.
  \end{theorem}
  
  \begin{proof}
    Suppose that $a \equiv b \pmod{b}$. Then, from the definition of congruence, $m \divides (a - b)$. From the definition of
    divisibility, we know that since there is an integer $k$ such that $km = a - b$, then $a = b + km$.
    
    Conversely, suppose that $a = b + km$. Then $a - b = km$, which means that $(a - b)$ is divisible by $m$. But that is exactly the definition
    of $a \equiv b \pmod{b}$.
  \end{proof}
  
  \begin{theorem}
    Every integer is congruent (mod $m$) to exactly one of $0, 1, \dots, m - 1$.
  \end{theorem}
  
  \begin{proof}
    Write $a = qm + r$, with $0 \leq r < m$. We know from Theorem \ref{th:division_algorithm} that $q$ and $r$ are uniquely determined.
    Since $a \equiv r \pmod{m}$, the theorem is proved.
  \end{proof}
  
  \begin{definition}
    The number $r$ in the last theorem is called the \textbf{least residue} of $a$ (mod $m$).
  \end{definition}
  
  \begin{theorem}
    $a \equiv b \pmod{m}$ if and only if $a$ and $b$ leave the same remainder on division by $m$.
  \end{theorem}
  
  \begin{proof}
    If $a$ and $b$ leave the same remainder $r$ when divided by $m$, then
    
    \begin{equation*}
      a = q_1m + r \;\;\text{and}\;\; b = q_2m + r
    \end{equation*}
    
    for some integers $q_1$ and $q_2$. It follows that
    
    \begin{equation*}
      a - b = (q_1m + r) - (q_2m + r) = m(q_1 - q_2)
    \end{equation*}
    
    From the definition of divisibility, we have $m \divides (a - b)$. From the definition of congruence, we conclude that
    $a \equiv b \pmod{m}$. To prove the converse, suppose that $a \equiv b \pmod{m}$. Then
    $a \equiv b \equiv r \pmod{m}$, where $r$ is a least residue modulo $m$. Then from Theorem 1,
    
    \begin{equation*}
      a = q_1m + r \;\;\text{and}\;\; b = q_2m + r
    \end{equation*}
    
    for some integers $q_1$ and $q_2$; since $0 \leq r < m$, the theorem is proved.
  \end{proof}
  
  \begin{lemma}
    For integers $a$, $b$, $c$, and $d$
    \begin{enumerate}[label=\alph*)]
      \item $a \equiv a \pmod{m}$.
      \item If $a \equiv b \pmod{m}$, then $b \equiv a \pmod{m}$.
      \item If $a \equiv b \pmod{m}$ and $b \equiv c \pmod{m}$, then $a \equiv c \pmod{m}$.
      \item If $a \equiv b \pmod{m}$ and $c \equiv d \pmod{m}$, then $a + c \equiv b + d \pmod{m}$.
      \item If $a \equiv b \pmod{m}$ and $c \equiv d \pmod{m}$, then $ac \equiv bd \pmod{m}$.
    \end{enumerate}
  \end{lemma}
  
  \begin{proof}
    ~
  \end{proof}
  
  \begin{enumerate}[label=\alph*)]
    \item $a - a = 0 = m \cdot 0$, so $m \divides (a - a)$, which means $a \equiv a \pmod{m}$.
    \item We are given $a - b = km$, so $b - a = (-k)m$, which means $b \equiv a \pmod{m}$.
    \item The first condition means that $a$ and $b$ leaves the same remainder $r_1$ when divided by $m$.
      The second means that $b$ and $c$ leaves the same remainder $r_2$ when divided by $m$.
      Because both $r_1$ and $r_2$ are the remainders when $b$ is divided by $m$, $r_1 = r_2$.
      Thus $a$ and $c$ has the same remainder when divided by $m$, or $a \equiv c \pmod{m}$.
    \item We are given that $a - b = km$ and $c - d = jm$ for some integers $k$ and $j$; thus
    
    \begin{align*}
      (a + c) - (b + d) &= (a - b) + (c - d) \\
        &= km - jm \\
        &= m(k - j)
    \end{align*}
    
    from the definition of congruence, $a + c \equiv b + d \pmod{m}$.
    
    \item We are given that $b - a = km$ and $d - c = jm$ for some integers $k$ and $j$; thus
    
    \begin{align*}
      ac - bd &= ac - (a + km)(c + jm) \\
        &= ac - ac - ajm - ckm - kjm^2 \\
        &= m (-aj -ck - kjm)
    \end{align*}
    
    from the definition of congruence, $ac \equiv bd \pmod{m}$.
  \end{enumerate}
  
  \begin{theorem}
    If $ac \equiv bc \pmod{m}$ and $(c, m) = 1$, then $a \equiv b \pmod{m}$.
  \end{theorem}
  
  \begin{proof}
    From the definition of congruence, $m \divides (ac - bc)$; consequently, $m \divides c(a - b)$.
    Because $(m, c) = 1$, we can conlude from Theorem 5 of Section 1 that $m \divides (a - b)$.
    That is, $a \equiv b \pmod{m}$.
  \end{proof}
  
  \begin{theorem}
    If $ac \equiv bc \pmod{m}$ and $(c, m) = d$, then $a \equiv b \pmod{m/d}$.
  \end{theorem}
  
  \begin{proof}
    If $ac \equiv bc \pmod{m}$, then $m \divides c(a - b)$ and $m/d \divides (c/d)(a - b)$.
    Since we know that $(m/d, c/d) = 1$, from Theorem 5 of Section 1 we get $m/d \divides (a - b)$,
    so $a \equiv b \pmod{m/d}$.
  \end{proof}
  
  \begin{theorem}
    Every integer is congruent $\pmod{9}$ to the sum of its digits.
  \end{theorem}
  
  \begin{proof}
    Take an integer $n$, and let its digital representation be

    \[ d_kd_{k-1}d_{k-2}\dots d_1d_0 \]

    That is,
    
    \begin{align*}
      n &= d_k10^k + d_{k-1}10^{k-1} + d_{k-2}10^{k-2} + \dots + d_1 10^1 + d_0 10^0 \\
        &\equiv d_k + d_{k-1} + d_{k-2} + \dots + d_0 \pmod{9}
    \end{align*}
  \end{proof}
  
  \section{Linear Congruences}
  
  \begin{definition}
    A solution to $ax \equiv b \pmod{m}$ is a number $r$ such that $ar \equiv b \pmod{m}$
    and $r$ is a least residue (mod $m$). There might be no solutions, exaction one solution,
    or many solutions.
  \end{definition}
  
  \begin{lemma}
    If $(a, m) \not\divides b$, then $ax \equiv b \pmod{m}$ has no solutions.
  \end{lemma}
  
  \begin{proof}
    We will prove the contrapositive, which is logically the same thing: if $ax \equiv b \pmod{m}$
    has a solution, then $(a, m) \divides b$. Suppose that $r$ is a solution. Then $ar \equiv b \pmod{m}$,
    and from the definition of congruence, $m \divides (ar - b)$, or from the definition of divides,
    $ar - b = km$ for some $k$. Since $(a, b) \divides a$ and $(a, b) \divides km$, it follows that
    $(a, m) \divides b$.
  \end{proof}
  
  \begin{lemma}
    If $(a, m) = 1$, then $ax \equiv b \pmod{m}$ has exactly one solution.
  \end{lemma}
  
  \begin{proof}
    Since $(a, m) = 1$, we know that there are integers $r$ and $s$ such that $ar + ms = 1$.
    Multiplying by $b$ gives
    
    \[ a(rb) + m(sb) = b \]
    
    Which means
    
    \[ a(rb) \equiv b \pmod{m} \]
    
    The least residue of $rb$ modulo $m$ is then a solution of the linear congruence.
    
    It remains to show that there is not more than one solution. Suppose that both $r$ and $s$
    are solutions. That is, since
    
    \[ ar \equiv b \pmod{m} \;\;\;\text{and}\;\;\; as \equiv b \pmod{m} \]
    
    Then $ar \equiv as \pmod{m}$. Because $(a, m) = 1$, we can apply Theorem 4 of the last section,
    cancel the common factor, and get $r \equiv s \pmod{m}$. That is, $m \divides (r - s)$. But $r$
    and $s$ are least residues (mod $m$), so
    
    \[ 0 \leq r < m \;\;\;and\;\;\; 0 \leq s < m \]
    
    Thus $-m < r - s < m$; together with $m \divides (r - s)$, this gives $r - s = 0$, or $r = s$, and
    the solution is unique.
    
  \end{proof}
  
\end{document}






