\documentclass{article}

\usepackage{amsmath}
\usepackage[utf8]{inputenc}
\usepackage{mathabx}
\usepackage{amsthm}
\usepackage{enumitem}

\newtheorem{theorem}{Theorem}[section]
\newtheorem{corollary}{Corollary}[theorem]
\newtheorem{lemma}[theorem]{Lemma}

\theoremstyle{definition}
\newtheorem{definition}{Definition}[section]

\title{Summary for ``Elementary Number Theory: Second Edition by Underwood Dudley"}
\date{2018-12-10}
\author{Agro Rachmatullah}

\begin{document}
  \pagenumbering{gobble}
  \maketitle
  
  \newpage
  \pagenumbering{arabic}
  \section{Integers}
  
  \begin{definition}
  $a$ divides $b$ (written $a \divides b$) if and only if there is an integer $d$ such that $ad = b$.
  
  \subparagraph{Examples} $3 \divides 6$, $15 \divides 60$, $9 \divides 9$, $-4 \divides 16$, and $2 \divides -100$.
  \end{definition}
  
  \begin{definition}
  If $a$ does not divide $b$, we write $a \notdivides b$.
  
  \subparagraph{Examples} $10 \notdivides 5$ and $3 \notdivides 7$.
  \end{definition}
  
  \begin{lemma}
    If $d \divides a$ and $d \divides b$, then $d \divides (a + b)$.
  \end{lemma}
  
  \subparagraph{Example} $2 \divides 4$ and $2 \divides 10$, so $2 \divides 12$.
  
  \begin{lemma}
    If $d \divides a_1$, $d \divides a_2$, ... $d \divides a_n$, then $d \divides (c_1a_1 + c_2a_2 + ... + c_na_n)$ for any integers $c_1, c_2, ..., c_n$
  \end{lemma}
  
  \subparagraph{Example} $2 \cdot 6 + 4 \cdot 9 = 12 + 36 = 48$. Because $3 \divides 6$ and $3 \divides 9$, we conclude that $3 \divides 48$.
  
  \begin{definition}
    $d$ is the greatest common divisor of $a$ and $b$ (written $d = (a, b)$) if and only if
    
    \begin{enumerate}[label=(\roman*)]
      \item $d \divides a$ and $d \divides b$, and
    
      \item if $c \divides a$ and $c \divides b$, then $c \le d$
    \end{enumerate}
  \end{definition}
  
  \subparagraph{Examples} $(2, 6) = 2$ and $(5, 7) = 1$.
\end{document}
