\documentclass{article}

\usepackage{amsmath}
\usepackage{mathabx}
\usepackage{amsthm}
\usepackage{enumitem}

\theoremstyle{definition}
\newtheorem{exercise}{Exercise}
\newtheorem{problem}{Problem}

\title{Exercises Solution for ``Elementary Number Theory: Second Edition by Underwood Dudley"}
\date{2018-12-10}
\author{Agro Rachmatullah}

\begin{document}
  \pagenumbering{gobble}
  \maketitle
  
  \newpage
  \pagenumbering{arabic}
  \section{Integers}
  
  \begin{exercise}
    Which integers divide zero?
  \end{exercise}
  
  For any integer $a$, $0 \cdot a = 0$. Therefore all integers divide zero.
  
  \begin{exercise}
    Show that if $a \divides b$ and $b \divides c$, then $a \divides c$.
  \end{exercise}
  
  From the definition, there are integers $d$ and $e$ such that $b = da$ and $c = eb$.
  Therefore,
  
  \begin{align*}
  c &= eb \\
     &= eda \\
     &= (ed)a
  \end{align*}
  
  Which means $a \divides c$.
  
  \begin{exercise}
    Prove that if $d \divides a$ then $d \divides ca$ for any integer $c$.
  \end{exercise}
  
  \subparagraph{Method 1} From the definition, there is an integer $b$ such that $a = bd$. Therefore $ca = cbd = (cb)d$
  which means $d \divides ca$.
  
  \subparagraph{Method 2} We can use Lemma 2 by setting $n = 1$, $a_1 = a$, and $c_1 = c$.
  
  \begin{exercise}
    What are $(4, 14)$, $(5, 15)$, and $(6, 16)$?
  \end{exercise}
  
  The positive divisors of 4 are 1, 2, and 4, and the positive divisors of 14 are 1, 2, 7, and 14. Therefore $(4, 14) = 2$.
  
  The positive divisors of 5 are 1 and 5. Likewise for 15 they are 1, 3, 5, and 15. Therefore $(5, 15) = 5$.
  
  The positive divisors of 6 are 1, 2, 3, and 6. For 16 they are 1, 2, 4, 8, and 16. Therefore $(6, 16) = 2$.
  
  \begin{exercise}
    What is $(n, 1)$, where $n$ is any positive integer? What is $(n, 0)$?
  \end{exercise}
  
  The only divisor of 1 is 1, and it also divides any positive integer $n$, so $(n, 1) = 1$.
  
  $n \divides n$ and is the largest divisor of $n$. Because $n \divides 0$, $(n, 0) = n$.
  
  \begin{exercise}
    If $d$ is a positive integer, what is $(d, nd)$?
  \end{exercise}

  The largest divisor of $d$ is $d$ itself. Because $d \divides nd$, $(d, nd) = d$.
  
  \begin{exercise}
    What are $q$ and $r$ if $a = 75$ and $b = 24$? If $a = 75$ and $b = 25$?
  \end{exercise}
  
  We can create the set
  
  \begin{equation*}
    \{75, 75 - 24 = 51, 75 - 2 \cdot 24 = 27, 75 - 3 \cdot 24 = 3\}
  \end{equation*}
  
  Therefore $75 = 3 \cdot 24 + 3$ so $q = 3$ and $r = 3$.
  
  Similarly, for the second problem we can create the set
  
    \begin{equation*}
      \{75, 75 - 25 = 50, 75 - 2 \cdot 25 = 25, 75 - 3 \cdot 25 = 0\}
    \end{equation*}
  
  So $q = 3$ and $r = 0$.
  
  \begin{exercise}
    Verify that the lemma is true when $a = 16$, $b = 6$, and $q = 2$.
  \end{exercise}
  
  We have the equation $16 = 6 \cdot 2 + 4$ so $r = 4$.
  
  $(16, 6) = 2$, and $(6, 4) = 2$, which is according to the lemma.
  
  \begin{exercise}
    Calculate $(343, 280)$ and $(578, 442)$.
  \end{exercise}
  
  For the first problem,
  
  \begin{align*}
    343 &= 280 + 63 \\
    280 &= 63 \cdot 4 + 28 \\
    63 &= 28 \cdot 2 + 7 \\
    28 &= 7 \cdot 4
  \end{align*}
  
  So $(343, 280) = (280, 63) = (63, 28) = (28, 7) = 7$
  
  For the second problem,
  
  \begin{align*}
    578 &= 442 + 136 \\
    442 &= 136 \cdot 3 + 34 \\
    136 &= 34 \cdot 4
  \end{align*}
  
  So $(578, 442) = (442, 136) = (136, 34) = 34$
  
  \begin{problem}
    Calculate $(314, 159)$ and $(4144, 7696)$.
  \end{problem}
  
  \begin{align*}
    314 &= 159 \cdot 1 + 155 \\
    159 &= 155 \cdot 1 + 4 \\
    155 &= 4 \cdot 38 + 3
  \end{align*}
  
  Therefore, using the Eucledian algorithm,
  
  \begin{align*}
    (314, 159) &= (159, 155) \\
    &= (155, 4) \\
    &= (4, 3) \\
    &= 1
  \end{align*}
  
  \begin{align*}
    7696 &= 4144 \cdot 1 + 3552 \\
    4144 &= 3552 \cdot 1 + 592 \\
    3522 &= 592 \cdot 6 + 0
  \end{align*}
  
  Therefore, using the Eucledian algorithm,
  
  \begin{align*}
    (4144, 7696) &= (7696, 4144) \\
    &= (4144, 3552) \\
    &= (3522, 592) \\
    &= 592
  \end{align*}
  
  \begin{problem}
    Calculate $(3141, 1592)$ and $(10001, 100083)$.
  \end{problem}
  
  \begin{align*}
    3141 &= 1592 \cdot 1 + 1549 \\
    1592 &= 1549 \cdot 1 + 43 \\
    1549 &= 43 \cdot 36 + 1
  \end{align*}
  
  Therefore, using the Eucledian algorithm,
  
  \begin{align*}
    (3141, 1592) &= (1592, 1549) \\
    &= (1549, 43) \\
    &= (43, 1) \\
    &= 1
  \end{align*}
  
  \begin{align*}
    100083 &= 10001 \cdot 10 + 73 \\
    10001 &= 73 \cdot 137 + 0
  \end{align*}
  
  Therefore, using the Eucledian algorithm,
  
  \begin{align*}
    (10001, 100083) &= (100083, 10001) \\
    &= (10001, 73) \\
    &= 73
  \end{align*}
  
  \begin{problem}
    Find $x$ and $y$ such that $314x + 159y = 1$.
  \end{problem}
  
  From problem 1, we know that a solution exists.
  
  \begin{alignat*}{5}
    314 &= 159 \cdot 1 + 155 &\;&\text{implies} &\;&&155 &= 314 - 159 \\
    159 &= 155 \cdot 1 + 4 &\;&\text{implies} &\;&&4 &= -314 + 159 \cdot 2 \\
    155 &= 4 \cdot 38 + 3 &\;&\text{implies} &\;&& 3 &= 314 \cdot 39 - 159 \cdot 77 \\
    4 &= 3 \cdot 1 + 1 &\;&\text{implies} &\;&& 1 &= 4 - 3
  \end{alignat*}
  
  Using backsubstitution we get
  
  \begin{align*}
    1 &= 314 (-40) + 159 \cdot 79
  \end{align*}
  
  So $x = -140$ and $y = 79$.
  
  \begin{problem}
    Find $x$ and $y$ such that $4144x + 7696y = 592$.
  \end{problem}
  
  From problem 1, we know that a solution exists.
  
  \begin{alignat*}{5}
    7696 &= 4144 \cdot 1 + 3552 &\;&\text{implies} &\;&&3552 &= 7696 - 4144 \\
    4144 &= 3552 \cdot 1 + 592 &\;&\text{implies} &\;&&592 &= 4144 - 3552
  \end{alignat*}
  
  Using backsubstitution we get
  
  \begin{align*}
    592 &= 7696 (-1) + 4144 \cdot 2
  \end{align*}
  
  So $x = -1$ and $y = 2$.
  
  \begin{problem}
    If $N = abc + 1$, prove that $(N, a) = (N, b) = (N, c) = 1$.
  \end{problem}
  
  Let $(N, a) = d$, so $d \divides N$ and $d \divides a$.
  
  Because $d \divides N$ and $d \divides abc$, from $N = abc + 1$ it must be the case that $d \divides 1$.
  
  Since $d$ is a gcd, $d \geq 1$, therefore $d = 1$.
  
  The same argument can be said for $(N, b)$ and $(N, c)$.
  
  \begin{problem}
    Find two different solutions of $299x + 247y = 13$.
  \end{problem}
  
  Using the Eucledian algorithm, we get $x_0 = 5$ and $y_0 = -6$.
  
  Since the original equation is a linear equation, it can be written as
  
  \begin{equation}
    y = -\frac{299}{247} x + c
  \end{equation}
  
  So from any point in the solution, we can go 247 units to the right
  and 299 units downwards and it will still be a solution. Therefore,
  
  \begin{align*}
    x &= 5 + 247n \\
    y &= -6 -299n
  \end{align*}
  
  Will be a solution for any integer $n$.
  
  \begin{problem}
  Prove that if $a \divides b$ and $b \divides a$, then $a = b$ or $a = -b$.
  \end{problem}
  
  From the proposition, $b = aq$ for some integer $q$ and $a = br$ for some integer $r$.
  Therefore,
  
  \begin{align*}
    a &= (aq)r \\
    &= a(qr)
  \end{align*}
  
  So $qr = 1$ which means $q = r = 1$ or $q = r = -1$, and because $a = br$ either
  $a = b$ or $a = -b$.
  
  \begin{problem}
    Prove that if $a \divides b$ and $a > 0$, then $(a, b) = a$.
  \end{problem}
  
  $a$ is the divisor of both $a$ and $b$. Because $a > 0$, $a$ is the largest divisor of $a$.
  So for any divisor $c$ of both $a$ and $b$, $c \leq a$. Thus $(a, b) = a$.
  
  \begin{problem}
    Prove that $((a, b), b) = (a, b)$.
  \end{problem}
  
  $(a, b) \divides b$ and of course $(a, b) \divides (a, b)$, so $(a, b)$ is a common divisor of both $(a, b)$ and $b$.
  Furthermore because $(a, b) > 0$, no other common divisor can be larger than it (else it won't divide $(a, b)$).
  Therefore $((a, b), b) = (a, b)$.
  
  \begin{problem}
    ~
  \end{problem}
  
  \begin{enumerate}[label=\alph*)]
      \item Prove that $(n, n + 1) = 1$ for all $n > 0$.
      
      Suppose that $(n, n + 1) = d$. It means that $d | n$ and $d | n + 1$, so it follows that $d | 1$.
  
      Because $d > 0$, then $d = 1$ (which is actually valid for all $n$).
      
      \item If $n > 0$, what can $(n, n + 2)$ be?
      
      Suppose $n$ is even, so $n = 2m$. Therefore
      
      \begin{equation*}
        (n, n + 2) = (2m, 2(m + 1))
      \end{equation*}
      
      Since $(m, m + 1) = 1$ as proved before, 2 is the largest common divisor of $2m$ and $2(m + 1)$.
      Therefore $(n, n + 2) = 2$, so if $n > 0$ then $(n, n + 2)$ could be 2.
  \end{enumerate}
    
  \begin{problem}
    ~
  \end{problem}
  
  \begin{enumerate}[label=\alph*)]
    \item Prove that $(k, n + k) = 1$ if and only if $(k, n) = 1$.
    
    If $d$ is a common divisor of both $k$ and $n + k$, $d \divides k$ and $d \divides n + k$, so it follows that $d \divides n$.
    
    However $(k, n + k) = 1$ so $d \leq 1$. Therefore $(k, n) = 1$. The reverse is trivially true.
    
    \item Is it true that $(k, n + k) = d$ if and only if $(k, n) = d$?
    
    Yes. Replace 1 in the argument above with any other number
  \end{enumerate}
  
  \begin{problem}
    Prove: If $a \divides b$ and $c \divides d$, then $ac \divides bd$.
  \end{problem}
  
  $a \divides b$ means $b = aq$ for some $q$, and $c \divides d$ means $d = cr$ for some $r$.
  
  Therefore
  
  \begin{align*}
    bd &= aqcr \\
    &= (ac)qr
  \end{align*}
  
  So $ac \divides bd$.
  
  \begin{problem}
    Prove: If $d \divides a$ and $d \divides b$, then $d^2 \divides ab$.
  \end{problem}
  
  This is just a more specific instance of the previous problem.
  
  \begin{problem}
    Prove: If $c \divides ab$ and $(c, a) = d$, then $c \divides db$.
  \end{problem}
  
  $a = dq$ for some $q$, and $c = dr$ for some $r$. In other words,
  $a/d = q$ and $c/d = r$. From Theorem 1.1, $(a/d, b/d) = 1$. Therefore
  $(q, r) = 1$.
  
  \begin{align*}
    a \divides ab &\implies dr \divides dqb \\
    &\implies r \divides qb
  \end{align*}
  
  Because $(q, r) = 1$, from Corollary 1.4.1 $r \divides b$, which means
  $dr \divides db$, or $c \divides db$.
  
  \begin{problem}
    ~
  \end{problem}
  
  \begin{enumerate}[label=\alph*)]
    \item If $x^2 + ax + b = 0$ has an integer root, show that it divides $b$.
    
    \begin{align*}
      x^2 + ax + b &= 0 \\
      b &= -x^2 - ax \\
      b &= x(-x - a)
    \end{align*}
    
    Which means that the root divides $b$.
    
    \item If $x^2 + ax + b = 0$ has a rational root, show that it is in fact an integer.
    
    Suppose that the roots are $r$ and $s$.
    
    \begin{align*}
      (x - r) (x - s) &= x^2 -(r + s)x + rs
    \end{align*}
    
    Therefore $a = -(r + s)$ and $b = rs$. 
    
    Let's say that one of the roots $r$
    can be written as $m/n$ where $(m, n) = 1$ (If it is not the case, then the rational number
    can be simplified so that $(m, n) = 1$). Assume that $m \neq 0$ because else the proof
    is done.
    
    Because $b = rs$, we have $s = \frac{bn}{m}$. Therefore
    
    \begin{align*}
      a &= -(r + s) \\
        &= -\left(\frac{m}{n} + \frac{bn}{m}\right) \\
        &= -\left(\frac{m^2 + bn^2}{mn}\right) \\
      -mna &= m^2 + bn^2 \\
      n(-ma) &= m^2 + n(bn)
    \end{align*}
    
    Because $n$ divides the left side and also $n(bn)$, we have $n \divides m^2$. We use
    the fact that $(m, n) = 1$ and Corollary 1.4.1 to conclude that $n \divides m$. Therefore
    $m/n$ is an integer.
    
  \end{enumerate}

\end{document}
