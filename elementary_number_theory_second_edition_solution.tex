\documentclass{article}

\usepackage{amsmath}
\usepackage{mathabx}

\title{Exercises Solution for ``Elementary Number Theory: Second Edition by Underwood Dudley"}
\date{2018-12-10}
\author{Agro Rachmatullah}

\begin{document}
  \pagenumbering{gobble}
  \maketitle
  
  \newpage
  \pagenumbering{arabic}
  \section{Integers}
  
  \paragraph{Exercise 1} Which integers divide zero?
  
  For any integer $a$, $0 \cdot a = 0$. Therefore all integers divide zero.
  
  \paragraph{Exercise 2} Show that if $a \divides b$ and $b \divides c$, then $a \divides c$.
  
  From the definition, there are integers $d$ and $e$ such that $b = da$ and $c = eb$.
  Therefore,
  
  \begin{align*}
  c &= eb \\
     &= eda \\
     &= (ed)a
  \end{align*}
  
  Which means $a \divides c$.
  
  \paragraph{Exercise 3} Prove that if $d \divides a$ then $d \divides ca$ for any integer $c$.
  
  \subparagraph{Method 1} From the definition, there is an integer $b$ such that $a = bd$. Therefore $ca = cbd = (cb)d$
  which means $d \divides ca$.
  
  \subparagraph{Method 2} We can use Lemma 2 by setting $n = 1$, $a_1 = a$, and $c_1 = c$.
\end{document}
